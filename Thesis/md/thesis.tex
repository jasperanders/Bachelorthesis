\hypertarget{which-requirements-need-to-be-matched-to-allow-for-sound-capability-testing-in-form-of-decentralized-e-exams}{%
\section{Which requirements need to be matched to allow for sound
capability testing, in form of decentralized
e-exams?}\label{which-requirements-need-to-be-matched-to-allow-for-sound-capability-testing-in-form-of-decentralized-e-exams}}

\hypertarget{introduction}{%
\subsection{Introduction}\label{introduction}}

The examination process is often a \textbf{tedious task} for those who
are in charge. Great amounts of time go into organizational problems.
Digitizing exams would resolve many of these problems. A step towards
electronic examination would make the process more flexible, scalable
and resource-efficient. Meanwhile, leading to a more accurate depiction
of a students' competence.

Taking the current pandemic under consideration, it may not be an option
to just move from paper to e-exams. Exams thus must be conducted
decentralized, i.d. students take their exams at home.

It is important to notice, that \emph{Decentralized E-exams (DE-exams)}
differ from \emph{paper based exams (PB-exams)} and even from
\emph{centralized e-exams (CE-exams)} in some key points. Foremost, the
examiner has less control over the environment the exam is taken under.
This raises questions about exam integrity and fairness. These questions
must be addressed through careful conceptualization of questions and
intelligent \protect\hyperlink{software}{software} design.

Digitizing exams is no new idea. Although, many concepts {[}cite papers
that talk about byod or multi media room exams{]} and implementations
focus on conducting e-exams in the same environment as \emph{PB-exams}.
With regard to COVID-19 this is not an option. Of course some
\emph{DE-exams} are \textbf{already conducted} today. These exams are,
for the most part, making use of a \textbf{proctoring system}. In such a
system a supervisor can access the examinees' device, can monitor all
their activity and will watch them through their webcam. This proctoring
process is costly. It hardly scales and still easily can be fooled.
Further, test-taking applications are found in many \emph{LMS' (Learn
Management Systems)} such as Ilias, Moodle or Blackboard. Unfortunately,
most often these applications focus on student self-assessment. They
also majorly \textbf{vary in quality and utility}. As they are
integrated in a complete LMS, changing to the \emph{best} implementation
is in many cases not an option. Last, as exam data is \textbf{highly
confidential}, there is a strong argument to be made against closed
source solutions. It is crucial to know exactly how the used application
works and how data is handled. Adding, open source projects are less
prone to major security issues as the development can leverage the
crowdsourcing capabilities that an open source system provides.

\hypertarget{an-overview-of-requirements}{%
\subsection{An overview of
requirements}\label{an-overview-of-requirements}}

We find e-exams to be advatageous in a variaty of ways. Still e-exams
are only valid if they can meet the same requirements that we are asked
for in paperbased exams. In his book, Handke provides suitable
requirements. The requirements are limited to topics on which the e-exam
software has direct influence (e.g.~requirements concerning exam content
are not being considerd). Further these requirements can be devided into
categories:

First, a requirement that defines the desired outcome of the exam:

\begin{itemize}
\tightlist
\item
  \textbf{General Validity}. Exams should aim to provide an accurate
  depiction of an examinees competence level.
\end{itemize}

Requirements that mainly influence interactions of examinees and
examiners with the examination system:

\begin{itemize}
\item
  \textbf{Protection against contestation.} No formal, or technical
  deficiencies should accure that would question the validity of the
  exam.
\item
  \textbf{Equal Treatment.} Individual examinees must be treated
  equally.
\item
  \textbf{Protection against cheating.} Exams must be protected against
  manipulation of exam outcome by examinees.
\item
  \textbf{Transparency.} The examination process and results must be
  verifiable.
\end{itemize}

Requirements that mainly influence the technical implementation of how
the examination system handels data:

\begin{itemize}
\item
  \textbf{Protection of Data.} Data of examinees is persoal data, as
  such it must be protected from missuse.
\item
  \textbf{Integrety.} Exam data must maintain consistency, accuracy and
  trustworthyness thoughout its entire lifetime.
\item
  \textbf{Attributability.} Taken exams must uniquely map to the
  examinee and vice versa.
\end{itemize}

Having identified these requirements, finding the matching design
principles is the next step.

\hypertarget{general-validity}{%
\subsubsection{General Validity}\label{general-validity}}

Matching design principle: Per question time constraints to allow for
partial open book exams.

As the states in their paper, examinations should support the purpose of
university to produce highly capable individuales. The degree to which
students succeed in that aspect is largely based on their performance in
exams. Subsequentually, students are highly incentivised to focus their
studies on the given exam format and questions. This interdependency
between knowledge aquisition and examination shows the importancy of
exam design and poses the question of what and how to test. E-exams
enable assessments to more closely depict a students actual skill level
by allowing the simultanieous of multiple question types.

Different question types are particularly well suited to test specific
aspects of learning. These questions types can be defined as follows :

\begin{itemize}
\item
  \textbf{(Semi) Closed questions}, which mainly revolve around the
  demonstration of \emph{factual knowledge}. Solutios are given by using
  a format like multipe choice (closed) or simple text input
  (semi-closed). \emph{e.g.~``What does \emph{BYOD} stand for?''}
\item
  \textbf{Competence questions}, which are suited to test for a certain
  \emph{practical skill}. Solutions are given in form of an
  implementation of the specific task at hand. \emph{e.g.~``Using the
  software, implement an e-exam about e-learning.''}
\item
  \textbf{Essay-type questions}, which are suited for assessing
  \emph{transfer knowledge} and understanding. Solutions are given by
  free text input. \emph{e.g.~``Explain why subjects in computer
  engineering are especially well suited for e-exams.''}
\end{itemize}

Further different degrees of allowed aid can be identified: In open book
exams, students are allowed to solve the question at hand using any
resource they feel they need. These open book exams rely mostly on both
competence and essay-type questions. It could be argued that these typs
of questions resemble a real world scenario in which access to
information is rarely limited. Meanwhile in such an open book exam
situation closed question are renderd insignificant as simple factual
knowledge is easly accessible. In order to ask closed questions it is
therefore necessary to restrict access to any aid.

Classical paper based exam do not provide a feasable way of combining
degrees of allowed aid. Therefore, some of the question groups tend to
be neglected. Resulting in constraints of possibilities to create an
accurate depicture of an examinees actual competence.

With e-exams on the other hand we can implement such a varying degree of
usable aid, creating a \emph{partial} open book exam. This can be
achived by letting students generally use any resource they need in
order to answer the question. Additionally we introduce per question
time constraints. These time constraints can be adjusted according to
the question type. Leaving closed questions with a harsh time
constraint, a \emph{either you know it or you don't} situation, where
there is no time to look up any solution. Essay-type questions just as
competence questions can use more generous timeframes. Leaving the
examinees room to make use of their tools.

Ultimatly, examination software in general has no direct impact of what
exact questions the examiner asks. The content of questions obviously
predifines how good a question can predict a examinees capabilities.
Still, with the use of patrial open book exams, e-exams allow for a
divers question set. Which allows to test factual, transfer and
practical knowledge to an equally vaild degree.

\hypertarget{protection-against-contestation}{%
\subsubsection{Protection against
contestation}\label{protection-against-contestation}}

Contestation of paper based exams generally is not a common problem.
This is mainly due to the controlled environment paper based exams are
taken under. Adding, the medium that is used to test examinees
(i.e.~paper) is failsave. E-exams, especially decentralized, ones
introduce the possibility of failure of the exam medium. E-exams rely on
software, on the operation of the exam device and of course on internet
connection.

The most crutial point probably is the reliability of the e-exam
software. This high reliability can only be achived by rigorous testing
and contious improvements. Of course, this applies to almost any
software artefact and thus may not be considered a specific design
principle per se. Further, device operatability largely lays in the
responsebility of the device owner. Still, examinees should be advised
to keep their devices updated, make sure they are working properly and
are plugged into power.

Softwarewise these directives can be supported by making the exam
accessible and workable if the internet connection is lost for short
periods of time. Additionally exam answers should contiously be send to
a server to minimize the possibility of data loss. In case of both a
device crash and internet failure, the exam should be persistent on the
local storage of the device. The device then can be rebooted and the
exam can be contiued.

\hypertarget{equal-treatment}{%
\subsubsection{Equal Treatment}\label{equal-treatment}}

Equal treatment of examinees should be carried out throughout the entire
examination process reaching from taking the exam to the correction of
the exam.

Possible inequality arises in some key areas. In BYOD exams student
devices are largely herogeneous, they run different operating systems
and consist of different hardware. This fakt should not lead to
different exam taking experiences. The choice of hardware should be
largely irrelavant. Consequentially, it makes little sense to develop
propriatary software for each operating system. Modern web technologies
provide a common language among different systems. Web applications do
not lack speed or functionallity and can be adopted cross platform. The
software is hosted at a central instance where it can be maintained and
improved and the software artefact is deliverd over a modern browser.

The process of correcting exams is also an area where possible
inequalities can be found. Especially in paper based exam checking an
exam for correctness is one of the most time-consuming processes in
conducting an exam. Resulting in fatigue and thus sometimes falts in
checking answers. Also a connection between bad hadwriting and worse
grades has been found. \textless{}James 1927\textgreater{} shows
students with bad handwriting get categorically worse grades than
students with better handwriting. By using e-exams these inequalities
can be fixed. First, some question types, such as multiple choice
questions can be checked automatically. This is an immediate improvement
over correcting these questions by hand. Second, as exam answers are
available in digital text reading and checking answers is easier.
Answers must not be deciphered, correction of exams can be done faster.
While also eliminating any biases against certain students.

\hypertarget{protection-against-cheating}{%
\subsubsection{Protection against
cheating}\label{protection-against-cheating}}

When thinking about any assessment the consideration and handeling of
academic dishonesty (e.g.~cheating in an exam) is one of the most
important parts. Moving from paper to e-exams poses the questions what
parts -- if any -- must be adjusted to accomodate for changed
circumstances and environments.

In his paper poses seven fields of possible cheating in exams which he
then evaluates by accurance and percived severeness. Six of which are
relevant for this thesisis purpose (The seventh would be \emph{Using
false excuse to delay taking test}). The fields are can be described as
follows:

\textbf{Student cooperation}:

\begin{itemize}
\tightlist
\item
  \textbf{Knowing the questions} Learning what is on an exam from
  someone who has already taken it.
\item
  \textbf{Cooperation with outsiders} Helping someone else cheat on an
  exam.
\item
  \textbf{Cooperation with fellow examinees} Copying from another
  student on an exam with their knowledge.
\end{itemize}

\textbf{Use of disallowed aid}:

\begin{itemize}
\tightlist
\item
  \textbf{Exploit environmental circumstances} Copying from another
  student on an exam without their knowledge.
\item
  \textbf{Use of unauthorised crib/cheat notes} Bringing prepared cheat
  notes to use in the exam.
\item
  \textbf{Use of electronic, unauthorized aid} Using e.g.~a smartphone
  to google or review lecture material.
\end{itemize}

Before thinking about how to obviate these cheating scenarios an
important statement must be made: Cheating cannot completely be
eliminated. There are always ways for students to engage in cheating.
E-exams cannot change that. But compared to paper based exams some
measures against cheating may be more effective.

\textbf{Knowing a question.} As the question finding process in a time
consuming process, a strategy may be to keep questions as secret as
possible and reuse them throughout multiple exams. This is a rather
uneffective strategie as platforms such as often provide comprehensive
protocols from memory of examinees who have engaged in a given exam.
E-exams can choose a different approach. Instead of having few questions
and keeping them secret, e-exams can levarage large question pools. At a
certain point it becomes unfeasable for students to prepare for every
available question as question pools grow larger. The digital nature of
these questions makes them easly shareable, allowing for larger question
pools more quickly.

\textbf{Cooperation with examinees.} For closed questions this
cooperation can be prevented by using strict time restictions. As
already stated above these questions fall in the category \emph{Either
you know the answer or you don't} there is no need for a lengthy
reflection period. As time constraints are tight, there really is no way
of communicating with others and solving the question. For more open
question types the time limitation is not as tight. At the same time
answers require more in depth considerations. To ensure that students
write down their own ideas and cannot share their thoughs, the copy and
paste functionallity can be disabled. Further, e-exams can easily be
randomized, thus preventing students form peeking or signaling solutions
to eachother.

\textbf{Cooperation with outsiders.} As decentralized e-exams are not
conducted in a controlled environment, cooperation with outsiders
becomes a bigger problem. Examinees could try to take the exam in the
presents of a more knowing person. Some try to solve this problem by
using proctored e-exams. These exams use live survaillance through
webcam and microphone that is evaluated by a person watching in
realtime. This apporach hardly scales as for every 4-5 students a
supervising proctor is needed. While programs like can make use of such
a system, their test fees of (??250€??) leave room for addidional
expenses. Although the live survaillance of stuents is not a valid
option the psycological effects of being monitored can be leveraged.

Thus a measure might be to make use of integrated webcams and
microphones of the devices at hand.This video and sound data can be
reviewed if needed. More importantly it creates a mental barrier. If
examinees really commit to engage in academic fraud they will most
certaily find a way to do so. The goal is simply to prevent those from
cheating that would cheat if they would feel no threat of being caught.
The sole existance of any measures makes the students behave more
honest. Just as video survailance makes crime less common at public
places .

\textbf{Exploit environmental circumstances.} Again randomization can
solve this problem. As questions appear in a different order for each
student even multiple choice questions can not simply be copyed.

\textbf{Use of unauthorized cheat notes or electronic aid.} Following
the argument made about partial open book exams we find that besides
time constraints no additional measures must be enforced. Cheat notes
really are not of much help if there is no time to use them.

We find these cheating scenarios to be to a large degree managed by
e-exams. Still, as specific software is in use, the degree of cheating
must constantly assessed. Measures against bugs or security flaws must
be identified.

\hypertarget{transparency}{%
\subsubsection{Transparency}\label{transparency}}

The examination process should be transparent for examinees. Examinees
must be able to understand their mistakes and shortcomings. This of
course implies that the exam software provides ways to give some kind
feedback. Second, as examiners are not free of mistakes, corrections can
sometimes be faulty. Well implemented transparency allows students to
review the examiners correction and possibly contest against singel
corrections. Important to mention is, that every student should get the
chance to review their exam. The digital nature of e-exams makes this
degree of transparency easly realizable. Sending a corrected digital
copy of an exam, lets examinees review their answers and understand
their gaps in knowledge. Contestance against singel questions could also
be processed in the exam software.

\hypertarget{attributability-protection-of-data-and-integrety}{%
\subsubsection{Attributability, Protection of Data and
Integrety}\label{attributability-protection-of-data-and-integrety}}

Exam data is highly sensetive and demands high levels of information
security. As with any information system, basic information security
principles apply. The following points prove to be of special
importance.

Exam data must be uniquely tracable to examinees. This can be easly
realized by having examinees log into an user account before they can
perform any action. Examinees either get a unique identifier in-software
or a unique identifier that is provided by the testing authotity.
Anything they do is then linked to their user id.

To assure solid data protection, strong user rights management must be
enacted to assure that only authorzied groups can see or correct exams.
Data in this way is largely protected from missuse. This measure ties
into the intergrety of exam data. As access is restricted exam data
cannot be changed. To provide even more security anserwed questions can
be send to a central server instance as soon as students start answering
the next question. Further frequent database backups of the exam data
should be standard procedure.

Another consideration to take into account is the availabiltiy of the
softwares source code. Processes should be completely transparent and
comprehensible. Exam authorties should be able to host exams by them
selves. This can be achived by providing the exam providing software in
open source format. This has the addintional benefit that such software
can benefit of crowd particpation.

\hypertarget{software}{%
\subsection{Software}\label{software}}

In the previous section I laid out areas in which e-exams can improve
upon the examination pocess. As I already discussed the actual accurance
of these advantages havily depends upon the software that is used in
order to implement an e-exam. In the following I will take a look at
common solutions:

\begin{itemize}
\tightlist
\item
  Ilias 
\item
  Moodle 
\item
  Blackboard 
\item
  LPlus 
\item
  OpenOlat 
\end{itemize}

I will measure their quality based on their degree of fullfillment of
requirements I layed out earlier.

Befor comparing these solutions there are some features that are met by
every tool. Among these are:

\begin{itemize}
\tightlist
\item
  A way to im-/export question and to manage question pools.
\item
  A way to implement all common question types.
\item
  A way to automatically grade closed questions.
\item
  A way to randomize the order of exam questions.
\item
  A in depth documentation and/or community.
\item
  Finished exams can be collected remotely.
\item
  Exam data is stored at a central instance and can be accessed
  remotely.
\item
  Examination is not bound to a singe location and issued over browser.
\end{itemize}

\hypertarget{learn-management-system-vs.-standalone-solution}{%
\subsubsection{Learn Management System vs.~Standalone
Solution}\label{learn-management-system-vs.-standalone-solution}}

Of the five e-exam products four are integrated into full LMS. LPlus
being the exception. The degree of usability and richness, highly varies
between the tools. Problematically the integration into a LMS makes them
unfeasable to exchange. As LMS provide a wide range of features and
applications the lock in effect a university experiences is way to large
as it could justify changing to another LMS, just to improve testing
capabilities. With this consideration in mind stand alone solutions have
an advantage. Instead of trying to solve a whole range of e-learnig
problems, a dedicated e-examination software focuses to solve a single
problem. I will still compare the e-examination capabilities of big LMS
as they often are the sole point of contact to e-examination of a
university.

\hypertarget{open-source}{%
\subsubsection{Open Source}\label{open-source}}

In the context of education, large amounts of personal data are
generated. Data protection and security play a big role in such an
environment.

Of the considerd solutions three of them are fully open source.
Blackboard only has a partially open source codebase and LPlus is a
closed source software.

\hypertarget{timing}{%
\subsubsection{Timing}\label{timing}}

As I introduced the concept of partial open book exams I showed how
important per question time restrictions are. Although all tools allow
for time restrictions effecting the whole exam, time constraints
enforceable in LPlus and OpenOlat

\hypertarget{connection-issues}{%
\subsubsection{Connection issues}\label{connection-issues}}

One of the biggest shortcomings of the softwares at hand is their way of
handeling connection errors. All of them rely on a stable internet
connection. Especially with questions that rely on time restrictions,
this is problematic. Student answers can be sent to the examination
authority after the time limt has expired due to delay in connectivity
although a student has answered the question in time.

\hypertarget{test-layout}{%
\subsubsection{Test layout}\label{test-layout}}

As described in Section \textless{}Sth.\textgreater{} exams must be
customizabel in such a way, that students cannot jump between questions,
cannot reanswer questions and can only see one question at a the time.
This customization is found in Moodle, LPlus and OpenOlat. Both Ilias
and Blackboard are very limited in that regard.

\hypertarget{checking-identity}{%
\subsubsection{Checking Identity}\label{checking-identity}}

In decentralized exams it is not trivial to check a students identity. I
disscussed possible measures in section \textless{}Sth.\textgreater{}.
None of the tested tools allow for any identity testing beyond
authentication via password. In a way this disqualifies all tools.

================================================================================

================================================================================

\hypertarget{the-case-for-e-exams}{%
\subsection{The case for e-exams}\label{the-case-for-e-exams}}

E-education is a much discussed topic. Most of the educational material
has become digital. Still, paper based exams are the way to go, when it
comes to assessments in German higher education. Altought some
universities amonge these are the FU Berlin and the Johannes Gutenberg
University Mainz have implemented some way of e-assessment, it still
lacks wide application.

In its research question this thesis ask about how to imporve upon
centralized e-exams. This implies, that e-exams in general are superior
to paper based exams. There is good reason for this implication. It is a
good idea to take a look at these reasons before moving on.

\hypertarget{why-move-from-paper-based-to-electronic-exams-in-the-first-place-inherented-vs.-earned-advantages}{%
\subsubsection{Why move from paper based to electronic exams in the
first place? inherented vs.~earned
advantages}\label{why-move-from-paper-based-to-electronic-exams-in-the-first-place-inherented-vs.-earned-advantages}}

Put simply central e-exams replace the pen and paper of paper based
exams with a computer or tablet. To give intuition of the advantage of
e-exams i will focus on \emph{bring-your-own-device} (BYOD) e-exams. In
BYOD exams the device the student is taking the exam on is not provided
by the universiyt authority, students bring their own tablet or laptop
to take the exam on. The BYOD concept is not new. For example Robert
Peregoodof talked about the BYOD implementation of the University of
British Columbia in the conference.

It is important to note that moving from paper based exams to e-exams
should be a strict uprgrade. Still, paper based exams and e-exams are
different in some key points. Some things that were considerated best
practice must be reevaluated and rethought in order to adequately fit
the e-exam context. As I will show these changes do not limit the
examination process but rather improve it.

When compared to paper based exams e-exams provide many advantages as
disscusses in their book. I will devide these into two different types.
The fist type are inheren t advantages. These are advantages that arise
from soley digitizing a paper based exam. These adavantages mainly
effect the efficiency of the assessment process.

The other type of advantages are created through additional
considerations and by rethinking how paper based exams approach the
examination process. These advantages arise mainly from improvements in
testing accuracy and equality.

In order to later assess the quality of existing software solutions,
after each section I will derive requirements. These requirements are
needed to make use of the theoretical advantage.

\hypertarget{exams-as-a-logistics-problem}{%
\subsubsection{Exams as a logistics
problem}\label{exams-as-a-logistics-problem}}

Thinking about the inherent advantages of e-exams, the logistical
implications of exams pop into mind. To illustrate this we use the
examination process at the KIT as an example . Although some steps may
differ from university to university, the gist remains.

Exams must be printed and stapled. As, exam taking students are
numerous, it is common for exams to take place at many different sites.
Therefore, on the test day exams must be carried out to the test site.
Here, the Ciw of the KIT recommends at least two supervisors per test
site. On test site, exams must be distributed to students. After the
exam is written, exams are collected and counted. They are then carried
back to a central location, where they remain until correction. For
answer checking, correctors come together, again at a central location,
where they then are able to correct the exam. After correction a grade
for students is published via internet. Succeeding, an exam revision for
students is planned and executed. Lastly exams are archived in their
paper form.

It is not supprising that handeling large amounts of paper results in
logistics overhead. Removing the paper, subsequentially removes much of
the logistics overhead. In an e-exam all of the exam data is digital.
There is no printing, since exams are directly transferred to the
students device. Further stundets answers can automatically be retrieved
after the exam is over. Exam answers are digitally available. The
correction of answers is no longer bound to a certain site but can be
done remotely. Exam results and feedback can be directly issued to the
students. Thus, revision can also be realized decentrally. The digital
exam data can then be easly archived.

This comparison illustrates the advantage of e-exams with regard to
logistics. Not only is there less movement of employees but more
importantly there is no movement of physical paper. Further we find
advantages in the digital archiving. Data can be stored space
efficiantly with no need for large physical archives. Adding, archives
are much saver as redundant backups are feasable and cheap.

\textbf{Requirements:}

\begin{itemize}
\tightlist
\item
  Finished exams can be collected remotely.
\item
  Exam data is stored at a central instance and can be accessed
  remotely.
\item
  Exam data can be arcived.
\end{itemize}

\hypertarget{statistics-on-the-fly}{%
\subsubsection{Statistics on the fly}\label{statistics-on-the-fly}}

Thinking of the digital nature of exam data another advantage emerges.
The digital nature of e-exams allow for a fast creation of statistics.
Whereas in paper exams every piece of information must be manually
digitized, e-exams are digital out of the box. Thus, analasys of exams
becomes more feasible. Thinkable are statistics about general
performance, but also analasys of specific questions, or student groups.
As the exam is the sole indicator of a students understanding of the
matter at hand, it is of utmost importance to understand where students
struggle and what they are capabil of. Having easy access to exam data
yields the possibility of both better exams and better courses.

\textbf{Requirements:}

\begin{itemize}
\tightlist
\item
  Possibility to create a statiscs automatically.
\end{itemize}

\hypertarget{questionpools}{%
\subsubsection{Questionpools}\label{questionpools}}

Part of the complexity and time intensiveness of exam creation lays in
creating appropriate questions. Although many courses are not unique to
one university, sharing of test question is not common. In paperbased
exams there are no standards and there is no suitable collaboration
platform. In paper based examination there is no real foundation for
sharing and reusing exam questions.

E-exams have to make use of an software artefact in oder to leverage
their theoretical benefits. Such a artefact allows for an enforcment of
a shareable general format. If users create exams in a specific
software, implementing a standard is fairly straight foreward.

\begin{quote}
Such a standard already exits under the name of \textless{}QTI
2.2\textgreater{}. Having a standard allowes for educators to
collaborate to create exams.
\end{quote}

Further this collaboration must not be limited by a singel exam with
only a few questions. Exams can come from question pools --- a small
subset of questions is selected from a way larger superset. The
development and maintenance of a large question pool is very time
intensive for a single person, that is why collaboration is so
important. Projects in open source software show how collaboration of
many can function with great success.

Large question pools do not only help in exam creation, as they allow to
produce exams automatically, it also prevents students from knowing the
exam beforhand. This will be further ellaborated in the section about
cheating.

\textbf{Requirements:}

\begin{itemize}
\tightlist
\item
  The exam software allows for management of question pools.
\item
  The exam software allows for export of questions, ideally for in a
  sharable format.
\end{itemize}

\hypertarget{from-central-to-decentral-e-exams}{%
\subsection{From central to decentral
e-exams}\label{from-central-to-decentral-e-exams}}

In the previous section I talked about advantages of central e-exams
compared to paper based exam. One key characteristic of e-exams is that
they are issued via the internet. We can use this to remove any local
component of e-exams, thus making them decentalized. In the following, I
will talk about advantages that decentral e-exams bare and the
implications of decentralizing exams on cheating and infrastructural
considerations.

\hypertarget{e-exams-as-an-logistics-problem}{%
\subsubsection{E-Exams as an logistics
problem}\label{e-exams-as-an-logistics-problem}}

As seen above, e-exams can be immensely more efficient in terms of
logistics, compared to a paper based exam. There are however still some
shortcomings. First, consider the allocation of students to their
respective test taking location. As different students take part in
different exams this allocation grows ever more complex and requires
lots of planning of a central authority. Making e-exams decentral allows
for this process to be streamlined enourmously.

Second, for supervision of exams human resources are needed. The for
example recommends two supervisors per location. E-exams can free up
some of these resources. To do this we need to take earlier
considerations about partial open book exams into account. As I layed
out, Partial open book exams allow for tests that do not ask for
supervision, making existing supervison obsolete.

When talking about decentralizing e-exams I think it is important to say
that the goal is not to force students to write their exams at home.
There still can be space reserved for exams by the university. The
managemant and supervision of these spaces just becomes magnitudes more
easy.

\textbf{Requirements:}

\begin{itemize}
\tightlist
\item
  Examination is not bound to a singe location.
\end{itemize}

\hypertarget{cheating}{%
\subsubsection{Cheating}\label{cheating}}
